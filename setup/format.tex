
% Standard packages

\usepackage[english]{babel}
% \usepackage[latin1]{inputenc}
\usepackage{times}
\usepackage[T1]{fontenc}
\usepackage{tabularx,multirow,multicol,keystroke,subfigure,longtable}

% 定制代码输出
\definecolor{lstbgcolor}{rgb}{0.9,0.9,0.9}
\usepackage{listings} %代码样式
\usepackage{fancyvrb}
\usepackage{xcolor}

%% 所要粘贴代码的编程语言
\lstloadlanguages{{[LaTeX]TeX}, {[ISO]C++}, {Java}, {Ruby}, {Scala}}

%% 设置listings宏包的一些全局样式
%% 参考http://hi.baidu.com/shawpinlee/blog/item/9ec431cbae28e41cbe09e6e4.html
\lstset{
escapeinside=`',
frameround=ftft,
keywordstyle=\color{blue!70},
commentstyle=\color{red!50!green!50!blue!50},
frame=shadowbox,
backgroundcolor=\color{yellow!20},
rulesepcolor=\color{red!20!green!20!blue!20}
numberbychapter=true,
breakatwhitespace=true,
showstringspaces=false,              %% 设定是否显示代码之间的空格符号
basicstyle=\scriptsize\ttfamily,           %% 设定字体大小\tiny, \small, \Large等等
commentstyle=\color{red!50!green!50!blue!50},                           
escapechar=`,                        %% 中文逃逸字符,用于中英混排
xleftmargin=1.5em,xrightmargin=0em, aboveskip=1em,
breaklines,                          %% 这条命令可以让LaTeX自动将长的代码行换行排版
extendedchars=false,                 %% 这一条命令可以解决代码跨页时,章节标题,页眉等汉字不显示的问题
frameround=fttt,
captionpos=top,
belowcaptionskip=1em
}

\lstdefinestyle{numbers}{
   numbers=left,
   numberstyle=\tiny,
   stepnumber=1,
   numbersep=1em
}

\lstdefinestyle{C++}{
   language=C++,
   texcl=true,
   prebreak=\textbackslash,
   breakindent=1em,
   % keywordstyle=\bfseries, %% 关键字高亮
   morekeywords={alignas, alignof, char16_t, char32_t, constexpr, decltype, noexcept, nullptr, static_assert, thread_local, OVERRIDE, INTERFACE, ABSTRACT, ROLE, USE_ROLE, IMPL_ROLE, __def_transaction,__def,__as,__as_procedure,__as_prot_procedure,__params,__apply,__with,
   __transaction,__procedure, __prot_procedure, __throw, __fix, __sync,__asyn,
   __concurrent,__sequential,__optional,__switch,__case, __loop0, __loop1,
   __timer_prot,__sleep,
   __fork,__join,__join_all,__signal,__sig_wait,__sig_sync,__broadcast,__sig_reply,
   __with_id,__finally,__on_fail,
   __call,__req,__rsp,__ind,__not,
   __wait,__peek,
   __action, __timer_id, __action_id, FT_FIXTURE, FT_SETUP, FT_TEARDOWN, TEST_F}
   %frame=leftline,                     %% 给代码加框
   %framerule=2pt,
   %rulesep=5pt
}

\lstnewenvironment{c++}[1][]
  {\lstset{style=C++, #1}}
  {}


\lstdefinestyle{Scala}{
   language=Scala,
   texcl=true,
   prebreak=\textbackslash,
   breakindent=1em,
   %frame=leftline,                     %% 给代码加框
   %framerule=2pt,
   %rulesep=5pt
}

\lstnewenvironment{scala}[1][]
  {\lstset{style=Scala, #1}}
  {}


\lstdefinestyle{Java}{
   language=Java,
   texcl=true,
   prebreak=\textbackslash,
   breakindent=1em,
   %frame=leftline,                     %% 给代码加框
   %framerule=2pt,
   %rulesep=5pt
}

\lstnewenvironment{java}[1][]
  {\lstset{style=Java, #1}}
  {}

% CJK packages
\usepackage{xeCJK}
%\setCJKmainfont{KaiTi}
%\setCJKmainfont{Adobe Heiti Std}

%% 设置缺省中文字体
\setCJKmainfont[BoldFont={Adobe Heiti Std}, ItalicFont={Adobe Kaiti Std}]{Adobe Song Std}   %  FZBaoSongZ04
%% 设置中文无衬线字体
\setCJKsansfont[BoldFont={Adobe Heiti Std}, ItalicFont={Adobe Kaiti Std}]{Adobe Kaiti Std}  
%% 设置等宽字体
\setCJKmonofont{Adobe Heiti Std}                            
%\setCJKmonofont{Monaco}                            

%% 英文衬线字体
\setmainfont{Lucida Bright}                                  
%% 英文等宽字体
%\setmonofont{Courier}
\setmonofont{Monaco}                             
%\setmonofont{Consolas}                              
%% 英文无衬线字体
\setsansfont{Optima}                                   

%% 定义新字体
\setCJKfamilyfont{song}{Adobe Song Std}                     
\setCJKfamilyfont{kai}{Adobe Kaiti Std}
\setCJKfamilyfont{hei}{Adobe Heiti Std}
\setCJKfamilyfont{fangsong}{Adobe Song Std}
\setCJKfamilyfont{lisu}{LiShu}
\setCJKfamilyfont{youyuan}{Adobe Kaiti Std}

%%自定义英文字体
\newfontfamily\couriernew{Lucida Grande}
\newfontfamily\optima{Optima}
\newfontfamily\lucida{Lucida Bright}

% Graphix
\usepackage{graphicx}

% Setup TikZ
\usepackage{tikz}
\usetikzlibrary{shapes,snakes,trees,arrows,positioning,shadows,automata}

\usepackage{amsmath,amssymb}

% 定义罗马数字
\makeatletter
\newcommand{\rmnum}[1]{\romannumeral #1}
\newcommand{\Rmnum}[1]{\expandafter\@slowromancap\romannumeral #1@}
\makeatother

% 定义破折号
\newcommand{\pozhehao}{\kern0.3ex\rule[0.8ex]{2em}{0.1ex}\kern0.3ex}

% 中文图表
\renewcommand\figurename{图}
\renewcommand\tablename{表}

%%% Local Variables:
%%% mode: LaTeX
%%% TeX-master: "../ann.tex"
%%% End:

